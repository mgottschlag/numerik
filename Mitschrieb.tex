\documentclass[a4paper,10pt]{article}
%\documentclass[a4paper,10pt]{scrartcl}


\usepackage[utf8x]{inputenc}
\usepackage[ngerman]{babel}
\usepackage{tikz}
\usepackage{mdwlist}
\usepackage{amsmath}
\usepackage{amssymb}
\usepackage{amsthm}
\usepackage{setspace}
\usepackage{color}
\usepackage{thmbox}
\usepackage{verbatim}
\usepackage[
  bookmarks=true,
    % Lesezeichen erzeugen
  bookmarksopen=true,
    % Lesezeichen ausgeklappt
  bookmarksnumbered=true,
    % Anzeige der Kapitelzahlen am Anfang der Namen der Lesezeichen
  pdfstartpage=1,
    % Seite, welche automatisch geöffnet werden soll
    % praktisch, wenn z.B. im Inhaltsverzeichnis gestartet
    % werden soll oder eine Seite bearbeitet wird.
  %baseurl=http://www.server.de/dateiname.pdf,
    % URL des PDF-Dokuments (oder Hintergrundinformationen)
  pdftitle={Numerik},
    % Titel des PDF-Dokuments
  pdfauthor={Florian Münchbach},
    % Autor(Innen) des PDF-Dokuments
  %pdfsubject={Kurzbeschreibung als ein Satz},
    % Inhaltsbeschreibung des PDF-Dokuments
  %pdfkeywords={Stichwörter},
    % Stichwortangabe zum PDF-Dokument
  breaklinks=true,
    % ermöglicht einen Umbruch von URLs
  colorlinks=true,
    % Einfärbung von Links
  linkcolor=black,
    % Linkfarbe: schwarz
  anchorcolor=black,
    % Ankerfarbe: schwarz
  citecolor=black,
    % Literaturlinks: schwarz
  filecolor=black,
    % Links zu lokalen Dateien: schwarz
  menucolor=black,
    % Acrobat Menü Einträge: schwarz
  pagecolor=black,
    % Links zu anderen Seiten im Text: schwarz
  urlcolor=black
    % URL-Farbe: schwarz
]
{hyperref}



%\usepackage{geometry}
%Kopf- und Fußzeile
%\usepackage{fancyhdr}

%\pagestyle{fancy}
%\fancyhf{}
 
%\renewcommand{\headrulewidth}{0pt}
%Fußzeile links bzw. innen
%\fancyfoot[L]{}
%Fußzeile mittig (Seitennummer)
%\fancyfoot[C]{\thepage}
%Linie unten
%\renewcommand{\footrulewidth}{0.5pt}
% Title
%\geometry{a4paper,left=2cm,right=2cm, top=2cm, bottom=2cm} 


%Kopf- und Fußzeile
\usepackage{fancyhdr}
\fancyhf{}

\pagestyle{fancy}
%Kopfzeile mittig mit Kaptilname
\fancyhead[C]{\nouppercase{\leftmark}}
%Linie oben
\renewcommand{\headrulewidth}{0.5pt}

%Fußzeile links bzw. innen
\fancyfoot[L]{}
%Fußzeile mittig (Seitennummer)
\fancyfoot[C]{\thepage}
%Linie unten
% \renewcommand{\footrulewidth}{0.5pt}



%Aufnahme in das Inhaltsverzeichnis
\setcounter{tocdepth}{4} 
%Nummerierung vertiefen
\setcounter{secnumdepth}{4}   




\title{Numerik - Mitschrieb SS2012}


\newcommand{\norm}[1]{ \left\Vert #1 \right\Vert }

% Syntax: \newtheorem{Name}[Zählung]{Bezeichnung}[Gliederung]
\newtheorem[L]{satz}{Satz}[section]
\newtheorem{definition}{Definition}[section]
\newtheorem{lemma}{Lemma}[section]
\newtheorem{bsp}{Beispiel}[section]
\newtheorem[S]{beweis}{Beweis}
\newtheorem[S]{beh}{Behauptung}


\begin{document}
  \begin{titlepage}
    \thispagestyle{empty}

    \newcommand{\HRule}{\rule{\linewidth}{0.5mm}}

    \begin{center}
      % Title
      \HRule \\[0.4cm]
      { \huge \bfseries Numerik}\\[0.5cm]
      \textsc{\LARGE Mitschrieb aus dem SS2012}\\
      \HRule \\[0.5cm]

      \vfill

%       % Author and supervisor
%       \begin{minipage}[r]{\textwidth}
% 	\begin{flushright}
% 	  \large
% 	\end{flushright}
%       \end{minipage}\\[1.5cm]
%       \begin{minipage}[r]{\textwidth}
% 	\begin{flushright}
% 	\end{flushright}
%       \end{minipage}

      \vfill

      % Bottom of the page
      {\large \today}

    \end{center}
  \end{titlepage}
%Workarround für leere Seite....
\newpage
  \thispagestyle{empty}
  \qquad
\newpage

\tableofcontents\thispagestyle{empty}
\newpage

\section{Normen}
\subsection{Vektornormen}
Die Algorithmen der numerischen Mathematik liefern Näherungswerte $\tilde{x}$ für die exakten Größen x.
Um die Güte von $\tilde{x}$ zu bestimmen brauchen wir ein Maß für die Differenz $x - \tilde{x}$.
Das Konzept einer Norm liefert solch ein Maß.

\smallskip

Eine Norm auf dem $\mathbb{K}$ -Vektorraum V ($\mathbb{K} \in \{\mathbb{R}, \mathbb{C}\}$) ist eine Funktion \\$\norm{\cdot} : V \rightarrow [0,\infty[$, für die gilt:
                                                                                                                                                     \begin{enumerate}
                                                                                                                                                      \item $\norm{\cdot} = 0 \Leftrightarrow v = 0$
                                                                                                                                                      \item $\forall v \in V, \forall \alpha \in \mathbb{K} : \norm{\alpha \cdot v} = |\alpha| \cdot \norm{v}$
                                                                                                                                                      \item $\forall v, w \in V: \norm{v + w} \leq \norm{ v} + \norm{w}$
                                                                                                                                                     \end{enumerate}

\smallskip

\begin{bsp}[Verschiedene Normen]
\begin{enumerate}
 \item  $V = \mathbb{K}^n$ über $\mathbb{K}$
  p-Normen:
  $\norm{ v }_p := (\sum \limits_{k=1}^{n} |v_k|^p )^{\frac{1}{p}}, 1 \leq p < \infty$
  \item Euklidische Norm:
  s. LA II

 \item  Maximumsnorm:
  $\norm{v}_\infty := max\{|v_k|: k=1:n\}$
 \item  Maximums oder Supremumsnorm:
  $\lim\limits_{p \rightarrow \infty} \norm{v}_p = \norm{v}_\infty$
\end{enumerate}
\end{bsp}

Wegen $\{v^k\} \subset V : \lim\limits_{n \rightarrow ???} v^k = w $ gilt $ \lim\limits_{k \rightarrow \infty} \norm{v^ k - w} = 0 $.

\smallskip

\begin{satz}[Äquivalenz von Normen]
Alle Normen auf einem endlichdimensionalen Vektorraum V sind \\\textbf{äquivalent}, das heißt zu jedem Paar von Normen $ \norm{\cdot}, \norm{\cdot}_* $ auf V \\existieren Konstanten
$0 < m \leq M$, so dass gilt:\\
$\forall v \in V: m \cdot \norm{v} \leq \norm{v}_* \leq M \cdot \norm{v} $
\end{satz}

\begin{bsp}[Äquivalenz von Normen]
$\forall v \in \mathbb{K}^n:
\norm{v}_q \leq \norm{v}_p \leq n^{(\frac{1}{p} - \frac{1}{q})} \norm{v}_q$ mit 
$1 \leq p \leq q \leq \infty$
\end{bsp}


\subsection{Matrixnormen}
\begin{satz}[Matrixnorm]
Seien $\norm{\cdot}_* $ und $ \norm{\cdot}_\bullet $ Normen auf $\mathbb{K}^n$ bzw. $\mathbb{K}^m$.\\
Dann ist $$ \norm{\cdot} := \mathbb{K}^{ m \times n} \rightarrow [0, \infty [ $$ mit
                                                                                                                                                                     $$\norm{A} := \max \limits_{v \in \mathbb{K}^n \ {0}} { \frac{\norm{Av}_\bullet}{\norm{v}_*}} = \max\limits_{v \in \mathbb{K}^n, \norm{v}_* = 1} { {\norm{Av}_\bullet}}$$
eine Norm auf dem Vektornormen der $m \times n$-Matrizen.\\
Sie wird \textbf{Matrix- oder Operatornorm} genannt.
\end{satz}

\begin{definition}
 obiger Satz ist zugleich eine Definition.
\end{definition}

\smallskip

\begin{beweis}[des Gleichheitszeichens in obiger Gleichung]
$\frac{\norm{Av}_\bullet}{\norm{v}_*} = \norm{A} \frac{v}{\norm{v}_*}$\\
Betrachten wir $\norm{\frac{v}{\norm{v}_*}}_* = \frac{1}{\norm{v}_*} * \norm{v}_* = 1$
\end{beweis}

\smallskip

Anmerkung:\\
Eine Matrixnorm auf $\mathbb{K}^{m 
\times n}$ wird durch die gewählten Normen auf $\mathbb{K}^n$ und $\mathbb{K}^m$ induziert; sie hängen also von diesen ab.\\

\smallskip

Was beschreibt die Matrixnorm?\\
Die Matrixnorm beschreibt die maximale Streckung, die ein Matrix, die auf einen Vektor angewandt wird, verursachen kann.

\bigskip

\begin{lemma}
Seien $\norm{\cdot}_*$ und $\norm{\cdot}_\bullet$ Normen auf $\mathbb{K}^n$ bzw. $\mathbb{K}^m$, die für alle n und m definiert sind (z.B. die p-Normen).\\
Dann ist die induzierte Matrixnorm submultiplikativ, d.h. 
$$\forall A \in \mathbb{K}^{m \times n} \forall B \in \mathbb{K}^{n \times r} :
\norm{AB} \leq \norm{A} \norm{B}$$
(Anmerkung: s. Ähnlichkeit zur Dreiecksungleichung).

\bigskip

Außerdem gilt:\\
Wegen $\forall v \in \mathbb{K}^n: \norm{Av}_\bullet \leq \norm{A} \norm{v}_*$ gilt o.B.d.A nun auch $ v \neq 0 $.\\
Damit ist $\frac{\norm{Av}_\bullet}{\norm{v}_*} \leq \norm{A}) = max \frac{\norm{Aw}_\bullet }{\norm{w}_*}$.

\end{lemma}

\newpage

\subsection{Matrix-p-Norm}
Sei $A \in \mathbb{K}^{m \times n}$.\\
Dann ist $\norm{A}_p := \max\limits_{\norm{v}_p = 1} \norm{Av}_p$ für $ 1 \leq p \leq \infty$ und $\norm{I_n}_p = 1$ wobei $I_n \in \mathbb{K}^{n \times n}$ die Einheitsmatrix ist.\\

\subsubsection{Spaltensummennorm}
  Sei $p = 1$, dann ist 
  \begin{eqnarray*}
  A = (a^{(1)}, \ldots, a^{(n)}, a^{(i)}) \in \mathbb{K}^m \\
  \norm{Av}_1 &=& \norm{\sum\limits_{i = 1}^n {v_i a^{(i)}}}_11 \leq \sum\limits_{i = 1}^n \norm{v_i a^{(i)}}_1\\
    &=& \sum\limits_{i = 1} |v_i| \cdot \norm{a^{(i)}}_1 \leq \max\limits{1 \leq i\leq n} \norm{a^{(i)}}_1 \norm{v}_1\\
  \end{eqnarray*}
  $\Rightarrow \frac{\norm{Av}_1}{\norm{v}_1} \leq \max\limits{1 \leq i\leq n} \norm{a^{(i)}}_1\\
  \Rightarrow \norm{A} \leq \max\limits{1 \leq i\leq n} \norm{a^{(i)}}_1$

  \bigskip

  Sei $ j \in \{1, \ldots, n\},\quad \norm{a^{(j)}}_1 = \max\limits_{1 \leq i\leq n} \norm{a^{(i)}}_1$
  $e^{(j)} \in \mathbb{K}^n; $
  \begin{equation*}
  e^{(j)}_k=
  \left\{
  \begin{aligned}
  % a &= b + c \\
  % d &= e + f 
  1 &: j = k,\\
  0 &: sonst
  \end{aligned}
  \right.
  \end{equation*}
  $\norm{e^{(j)}}_1 = 1\\
  \norm{Ae^{(j)}}_1 = \norm{a^{(j)}}_1 = ( \max\limits{1 \leq i\leq n} \norm{a^{(i)}}_1) \cdot \norm{e^{(j)}}_1\\
  \Rightarrow \norm{A}_1 = \max\limits{1 \leq i\leq n} \norm{a^{(i)}}_1 = \max\limits_{1 \leq i \leq n} \sum\limits_{j = 1}^n{|a_{j,i}|}$

  \medskip

  Daher heißt die Matrix-1-Norm auch \textbf{Spaltensummennorm}.\\

  \bigskip

  \begin{bsp}
  Sei $A = 
  \begin{pmatrix}
  1 & 10 & 3\\
  -5 & -1 & 0\\
  3i & \sqrt{2} & 0
  \end{pmatrix}$\\
  Dann ist $\norm{A}_1 = \max\{9, 11 + \sqrt{2}, 3\} = 11 + \sqrt{2}$
  \end{bsp}

\subsubsection{Zeilensummennorm}
  Sei nun $p = \infty$, dann heißt
  $\quad \norm{A}_\infty = \max\limits_{1\leq i \leq m}\sum\limits_{j=1}^n | a_{i,j}| \quad$
  \textbf{Zeilensummennorm}.

\begin{bsp}
(Fortsetzung von obigem Beispiel)\\
 $\norm{A}_\infty = \max\{14, 6, 3+\sqrt{2}\} = 14$
\end{bsp}

\subsubsection{Spektralnorm}
Sei $ p = 2$, dann heißt die Matrix-2-Norm \textbf{Spektralnorm}, denn $$\norm{A}_2 = \lambda_{max} (A^H  A)^\frac{1}{2}$$
(größter Eigenwert von $A^H A$)\\
Eigenschaften:\\
\begin{itemize}
 \item $\norm{A}_2 = \norm{A^H}_2$
 \item $\norm{A^H A}_2 = \norm{A}_2^2$
 \item $\forall Q$ unitär : $\norm{Q A}_2 = \norm{A}_2$ \hfill unitär $Q^T Q = I_n, Q^H Q = I_n$
\end{itemize}

\subsection{Kondition einer Matrix}
Sei $A \in \mathbb{K}^{n\times n}$ regulär.
Dann heißt
$$\kappa (A) =\norm{A} \norm{A^{-1}}$$
die Kondition (Konditionszahl) von A bzw. $\norm{\cdot}$.\\
Die Konditionszahl ist ein Maß für die Sensitivität der Lösung eines linearen Gleichungssystems $Ax = b$ gegenüber Störungen in b (und auch in A).

\begin{satz}
 A sei regulär, dann gilt:
$$\frac{\norm{A^{-1} b - A^{-1} \tilde{b}}}{\norm{A^{-1} b}} \leq \kappa(A) \cdot \frac{\norm{b - \tilde{b}}}{\norm{b}}$$ ist scharf.
Der Kondition $\kappa(A)$ liegt hier die von $\norm{\cdot}$ induzierte Matrixnorm zugrunde.
\end{satz}

\begin{beweis}
 $\frac{\norm{A^{-1} b - A^{-1} \tilde{b}}}{\norm{A^{-1} b}} = \norm{A^{-1} (b - \tilde{b})} \leq \norm{A^{-1} } \norm{b - \tilde{b}}\\
\norm{b} = \norm{AA^{⁻1} b} \leq \norm{A}\norm{A^{-1} b} \Rightarrow \frac{1}{\norm{A^{-1} b}} \leq \frac{\norm{A}}{\norm{b}}$\\
Wegen $a < b, c < d \Rightarrow ac < bd$ folgt: $$ \frac{\norm{A^{-1} b - A^{-1} \tilde{b}}}{\norm{A^{-1} b}} \leq \underbrace{\kappa(A)}\limits_{\norm{A} \norm{A^{-1}}} \frac{\norm{b - \tilde b}}{\norm{b}}\quad.$$\\\hfill \qed
\end{beweis}

\begin{beh}
 $\kappa(A) \geq 1\\
1 = \norm{I_n} = \norm{A \cdot A^{-1}} \leq \norm{A}\norm{A^{-1}} = \kappa(A)$
\end{beh}
Die Konditionszahl einer Matrix misst, wie weit diese von einer singulären (d.h. nichtregulären) Matrix entfernt ist.\\
\begin{satz}
 Sei $A \in \mathbb{K}^{n\times n}$ regulär. Dann ist
$$\min \left\{\frac{\norm{A - S}_p}{\norm{A}_p} : S \in \mathbb{K}~ singul\ddot{a}r\right\} = \frac{1}{\kappa(A)}$$\\
wobei $\kappa = \norm{A}_p \norm{A^{-1}}_p$.
\end{satz}

\begin{satz}
 Sei $\lambda_{\max{(A)}}$ der betragsgrößte Eigenwert von $A \in \mathbb{K}^{n \times n}$. Dann gilt\\ 
$$ | \lambda_{\max(A)}|  = \inf\left\{\norm{A} : \norm{\cdot} ist~induzierte~Matrixnorm\right\}$$\\
Das heißt, zu jedem $\delta > 0$ gibt es eine induzierte Matrixnorm $\norm{\cdot}_\delta$, so dass $$ \norm{A}_{\delta} y  | \lambda_{\max{(A)}} | + \delta$$
ist.
\end{satz}


$A \in \mathbb{K}^{n \times n}, \lambda \in \mathbb{C}$ ist Eigenwert von A.\\
Es gilt: $ | \lambda | \leq \norm{A}$, wobei $\norm{\cdot}$ induzierte Matrixnorm ist.
\newpage

\section{Lineare Gleichungssysteme: Direkte Löser}
Gegeben $A = \{a_{i,j}\} \in \mathbb{R}^n$.\\
Gesucht: $x \in \mathbb{R}^n$ mit $Ax = b$.
\begin{equation} \label{eq:lgs}
 Ax = b \Leftrightarrow \sum\limits_{j = 1}^n a_{i,j}x_j = b_j, 1 \leq i\leq n 
\end{equation}


\subsection{Auflösung gestaffelter Systeme}
Sei $Rx = z$ mit $r_{i,j} = 0, i > j$, dann ist R eine obere Dreiecksmatrix.\\

\smallskip
 \begin{eqnarray*} %use asterisk here to supress numbering  r_{1,1}x_1 + r_{1,2}x_2+ \ldots r_{1,n}x_n &=& z_1\\
  r_{2,2}x_2+ \ldots r_{2,n}x_n &=& z_2\\
  \vdots\\
  r_{1,n}x_n &=& z_n
 \end{eqnarray*}

Wegen $0 \neq \det{R} = \prod \limits_{i = 1}^n r_{i,i} $ folgt $ r_{i,i} \neq 0$ für $ 1 \leq i \leq n$.
\begin{eqnarray*}
  x_n &=& \frac{z_n}{r_{n,n}}\\
  x_{n-1} &=& \frac{  z_{n-1} - r_{n-1, n} x_n  }{  r_{n-1,n-1}  }\\
  \vdots\\
  x_{n-1} &=& \frac{z_{i} - r_{i, i+1} x_{i+1} - \ldots - r_{i, n} x_n}{r_{i,i}}\\
  \vdots\\
  x_{n-1} &=& \frac{z_{1} - r_{1, 2} x_{2} - \ldots - r_{1, n} x_n}{r_{1,1}}
\end{eqnarray*}

\texttt{(Rückwärtssubstitution)}\\

\smallskip

\textbf{Aufwand:} Berechnung von $x_i$ benötigt (n-i) Additionen und (n-i) Multiplikationen, so wie 1 Division.
$\Rightarrow \sum\limits_{i=1} {[2(n-i)+1 ]} = n²$ Operationen.\\

\smallskip

Das Auflösen eines unteren Dreieckssystems: $Lx = z$ mit $l_{i,j}= 0  , i< j$
verläuft analog und heißt \texttt{Vorwärtssubstitution}.

\subsection{Gaußscher Algorithmus: LR-Zerlegung}

Idee: Führe \eqref{eq:lgs} über in ein Dreieckssystem, das dieselbe Lösung besitzt.

Zulässige Umformungen sind:
\begin{itemize}
  \item Vertauschen von Zeilen
  \item Vertauschen von Spalten von $A$ (Umnummerierung der Unbekannten nötig)
  \item Multiplikation einer Zeile mit einer Zahl $\neq 0$
  \item Addition des Vielfachen einer Zeile zu einer anderen
\end{itemize}

Verkürzte Schreibweise des Gleichungssystems:

\[
  \begin{pmatrix}
    a_{1,1} & \dots & a_{1,n} & b_1 \\
    a_{2,1} & \dots & a_{2,n} & b_2 \\
    \vdots & & & \vdots \\
    a_{n,1} & \dots & a_{n,n} & b_n \\
  \end{pmatrix}
  = (A, b) = (A^{(1)}, b^{(1)})
\]

\textbf{1. Schritt:} Produziere Nullen unterhalb von $a_{1,1}$. Es sei $a_{1,1} \neq 0$.
  Subtrahiere das $l_{j,1} := \frac{a_{j,1}}{a_{1,1}}$-fache der 1. Zeile von der j-ten Zeile.

\[
  \begin{pmatrix}
    a_{1,1}^{(1)} & & \dots & a_{1,n}^{(1)} & b_1^{(1)} \\
    0 & a_{2,2}^{(2)} & \dots & a_{2,n}^{(2)} & b_2^{(2)} \\
    \vdots & \vdots & & \vdots \\
    0 & a_{n,2}^{(2)} & \dots & a_{n,n}^{(2)} & b_n^{(2)} \\
  \end{pmatrix}
  = (A^{(2)}, b^{(2)})
\]

Dabei ist
\begin{align*}
  a_{j, i}^{(2)} = a_{j, i}^{(1)} - l_{j, 1} a_{1, i}^{(1)} \\
  b_{j}^{(2)} = b_{j}^{(1)} - l_{j, 1} b_{1}^{(1)} \\
\end{align*}

Nach dem Schritt kann $A^{(2)}$ anders geschrieben werden als $A^{(2)} = L_1 A^{(1)}$, $b^{(2)}=L_1 b^{(1)}$ mit

% TODO: Matrix
\begin{comment}
\[
  L_1 = \begin{pmatrix}
    1 & & & & 0 \\
    -l_{2,1} & 1 \\
    -l_{3,1} & 0 & 1 \\
     \vdots & \vdots & & \ddots \\
     -l_{n,1} & 0 & 0 & \dots & 1 \\
     % TODO!
  \end{pmatrix}
\]
\end{comment}

\textbf{k-ter Schritt ($2 \leq k \leq n-1$):}

\[
  (A^{(k)}, b^{(k)}) \rightarrow (A^{(k+1)}, b^{(k+1)})
\]

% TODO: Matrix
\begin{comment}
\[
  \begin{pmatrix}
    a_{1, 1}^(1) & & & & \dots & a_{1, n}^(1) & b_1^{(1)} \\
    0 & a_% TODO{2, 2}^(2) & & & \dots & a_{2, n}^(2) & b_2^{(2)} \\
    0 & 0 & \ddots & & &  & \vdots & \vdots \\
    0 & 0 & 0 & a_{k, k}^(k) & \dots & a_{k, n}^(k) & b_k^{(k)} \\
    \vdots & \vdots & \vdots & \vdots & & \vdots & \vdots \\
    0 & 0 & 0 & a_{n, k}^(k) & \dots & a_{n, n}^(k) & b_n^{(k)} \\
  \end{pmatrix}
\]
\end{comment}

Sei $a_{k,k}^{(k)} \neq 0$.

\begin{align*}
  &l_{j, k} := a_{j,k}^{(k)} / a_{k,k}^{(k)}, j=k+1,\dots, n \\
  &a_{j,k}^{(k+1)} = a_{j,i}^{(k)} - l_{j, k} a_{k,i}^{(k)}, k+1 \leq i,j \leq n \\
  &b_{j}^{(k+1)} = b_{j}^{(k)} - l_{j, k} b_{k}^{(k)}, j=k+1,\dots, n \\
\end{align*}

$a_{k,k}^{(k)}$ heißt Pivot-Element der $k$-ten Spalte.

\[
  A^{k+1} = L_k A^{(k)}, b^{(k+1)} = L_k b^{(k)}
\]

\[
  L_k = \begin{pmatrix}
    TODO
  \end{pmatrix}
  = I_n - l_k \underbrace{{{e^{(k)}}^T}}_{\in \mathbb{R}^{n \times n}}
\]

\[
  l_k = \begin{pmatrix}
          0 \\ \vdots \\ 0 \\ l_{k+1,k} \\ \vdots \\ l_{n,k}
        \end{pmatrix},
  e^{(k)} = \begin{pmatrix} 0 \\ \vdots \\ 0 \\ 1 \\ 0 \\ \vdots \\ 0 \end{pmatrix} % TODO: k-te Komponente markieren
\]

Falls $a_{k,k}^{(k)} \neq 0$ für $k=1, \dots, n-1$, dann $R = A^{(k)} = L_{n-1} A^{(n-1)} = L_{n-1} L_{n-2} \dots L_{1} A$.

$\Rightarrow A = LR$ mit $L=L_1^{-1} L_2^{-1} \dots L_{n-1}^{-1}$

\textbf{Beh.:} $L_k^{-1} = I_n + l_k {e^{(k)}}^T$

\textbf{Bew.:} $L_k (I_n + l_k {e^{(k)}}^T) = I_n - l_k \underbrace{{e^{(k)}}^T l_k}_{=0} {e^{(k)}}^T$

\begin{align*}
  \Rightarrow L_k &= (I_n + l_1 {e^{(1)}}^T) (I_n + l_2 {e^{(2)}}^T) \dots (I_n + l_{n-1} {e^{(n-1)}}^T) \\
                  &= I_n + {e^{(1)}}^T + l_2 {e^{(2)}}^T + \dots l_{n-1} {e^{(n-1)}}^T \\
                  &= \begin{pmatrix}
                       1 \\
                       l_{2,1} & 1 & & 0 \\
                       l_{3,1} & l_{3,2} & 1 \\
                       \vdots & \vdots & \ddots & \ddots \\
                       l_{n,1} & l_{n,2} & \dots & l_{n,n-1} & 1 \\
                     \end{pmatrix} \\
\end{align*}

\begin{definition}
  Die Zerlegung einer Matrix $A = LR$ in eine obere Dreiecksmatrix $R$ und eine untere Dreiecksmatrix $L$, die nur Einsen auf de Diagonalen hat, heißt \textbf{Gaußsche Dreieckszerlegung} oder \textbf{LR-Zerlegung}.
\end{definition}

\begin{bsp}
  \[
    A = \begin{pmatrix}
          2 & 1 & 1 \\
          4 & 3 & 3 \\
          8 & 7 & 9
        \end{pmatrix}
  \]

  \[
    \underbrace{\begin{pmatrix}
      1 & 0 & 0 \\
      -2 & 1 & 0 \\
      -4 & 0 & 1
    \end{pmatrix}}_{L_1}
    \underbrace{\begin{pmatrix}
      2 & 1 & 1 \\
      4 & 3 & 3 \\
      8 & 7 & 9
    \end{pmatrix}}_{A^{(1)}}
    =
    \underbrace{\begin{pmatrix}
      2 & 1 & 1 \\
      0 & 1 & 1 \\
      0 & 3 & 5
    \end{pmatrix}}_{A^{(2)}}
  \]

  \[
    \underbrace{\begin{pmatrix}
      1 & 0 & 0 \\
      0 & 1 & 0 \\
      0 & -3 & 1
    \end{pmatrix}}_{L_2}
    \underbrace{\begin{pmatrix}
      2 & 1 & 1 \\
      0 & 1 & 1 \\
      0 & 3 & 5
    \end{pmatrix}}_{A^{(2)}}
    =
    \underbrace{\begin{pmatrix}
      2 & 1 & 1 \\
      0 & 1 & 1 \\
      0 & 0 & 2
    \end{pmatrix}}_{A^{(3)}=R}
  \]

  \[
    L = L_1^{-1} L_2^{-1} =
      \begin{pmatrix}
        1 & 0 & 0 \\
        2 & 1 & 0 \\
        4 & 3 & 1 \\
      \end{pmatrix}
  \]

  \[
    A =
      \begin{pmatrix}
        1 & 0 & 0 \\
        2 & 1 & 0 \\
        4 & 3 & 1 \\
      \end{pmatrix}
      \begin{pmatrix}
        2 & 1 & 1 \\
        0 & 1 & 1 \\
        0 & 0 & 2
      \end{pmatrix}
  \]

\end{bsp}

\subsection{Gauß-Elimination zur Lösung von (2.1)} % TODO: Subsection?

\begin{enumerate}
  \item $A = LR$ (LR-Zerlegung)
  \item $Lz = b$ Vorwärtssubstitution
  \item $Rx = Z$ Rückwärtssubstitution
\end{enumerate}

Die LR-Zerlegung benötigt dabei $O(^3)$ Operationen.

\subsection{LR-Zerlegung mit Spaltenpivotsuche}

Vertausche im $k$-ten Schritt (2.2) die $k$-te Zeile mit einer Zeile $j \geq k$, für die gilt

\[
  |a_{j,k}^{(k)} | = \max \limits_{k \leq i \leq n} | a_{i,k}^{(k)} |
\]

\begin{bsp}
  \[
    p = \begin{pmatrix} 1 & 2 & 3 \end{pmatrix}
  \]

  \begin{align*}
    &\begin{pmatrix}
       2 & 1 & 1 \\
       4 & 3 & 3 \\
       8 & 7 & 9
    \end{pmatrix} \\
  &\xrightarrow[p=(3\ 2\ 1)]{Pivot}
    \begin{pmatrix}
       8 & 7 & 9 \\
       4 & 3 & 3 \\
       2 & 1 & 1
    \end{pmatrix} \\
  &\stackrel{\textnormal{El.}}{\rightarrow}
    \begin{pmatrix}
       8 & 7 & 9 \\
       \frac{1}{2} & -\frac{1}{2} & -\frac{3}{2} \\
       \frac{1}{4} & -\frac{3}{4} & -\frac{5}{4}
    \end{pmatrix} \\
  &\xrightarrow[p=(3\ 1\ 2)]{Pivot}
    \begin{pmatrix}
       8 & 7 & 9 \\
       \frac{1}{4} & -\frac{3}{4} & -\frac{5}{4} \\
       \frac{1}{2} & -\frac{1}{2} & -\frac{3}{2}
    \end{pmatrix} \\
  &\stackrel{\textnormal{El.}}{\rightarrow}
    \begin{pmatrix}
       8 & 7 & 9 \\
       \frac{1}{4} & -\frac{3}{4} & -\frac{5}{4} \\
       \frac{1}{2} & \frac{2}{3} & -\frac{2}{3}
    \end{pmatrix}\\
  \end{align*}

  \[
    \Rightarrow
    \underbrace{\begin{pmatrix}
       0 & 0 & 1 \\
       1 & 0 & 0 \\
       0 & 1 & 0
    \end{pmatrix}}_P
    \underbrace{\begin{pmatrix}
       2 & 1 & 1 \\
       4 & 3 & 3 \\
       8 & 7 & 9
    \end{pmatrix}}_A
    =
    \underbrace{\begin{pmatrix}
       1 & 0 & 0 \\
       \frac{1}{4} & 1 & 0 \\
       \frac{1}{2} & \frac{2}{3} & 1 \\
    \end{pmatrix}}_L
    \underbrace{\begin{pmatrix}
       8 & 7 & 9 \\
       0 & -\frac{3}{4} & -\frac{5}{4} \\
       0 & 0 & -\frac{2}{3} \\
    \end{pmatrix}}_R
  \]
\end{bsp}

\begin{satz}
  Der Gaußalgorithmus mit Spaltenpivotsuche liefert für jede reguläre Matrix $A$ eine Permutationsmatrix $P$ sowie Dreiecksmatrizen $L$ und $R$, so dass
  \[
    PA = LR
  \]
  die LR-Zerlegung von $PA$ ist. Die Elemente von $L$ sind betragsmäßig kleiner oder gleich 1.
\end{satz}

\textbf{Bemerkung:}
\[
  A = PLR \Rightarrow \det A = \det P \underbrace{\det L}_{=1} \det R = \det P \prod_{i=1}^n r_{i,i}
\]

\[
  \det P = (-1)^l, l\textnormal{ Anzahl der Zeilenvertauschungen}
\]

\end{document}
